\documentclass[twocolumn]{article}
\usepackage[utf8]{inputenc}
\usepackage{amsmath}
\setcounter{secnumdepth}{0}
\setcounter{MaxMatrixCols}{20}
\usepackage{graphicx}
\usepackage{float}
\usepackage{caption}
\usepackage{subfig}
\usepackage{hyperref}
\graphicspath{ {./images/} }  

\title{Formulario, Física}
\author{Raúl Ultralaser}
\date{}

\begin{document}

\maketitle

\section{Vectores}
\begin{align*}
\vec{A}&=A_x\hat{i}+A_y\hat{j}+A_z\hat{k}\\
\vec{B}&=B_x\hat{i}+B_y\hat{j}+B_z\hat{k}\\
\end{align*}
Producto escalar (punto)
\begin{equation*}
    \Vec{A}\cdot \Vec{B}=ABcos\phi=|A||B|cos\phi
\end{equation*}
en terminos de sus componentes
\begin{equation*}
    \Vec{A}\cdot \Vec{B}=A_xB_x+A_yB_y+A_zB_z
\end{equation*}
Producto cruz
\begin{equation*}
    \Vec{C}=\Vec{A}\times\Vec{B}=ABsen\phi
\end{equation*}
en terminos de sus componentes
\begin{align*}
C_x&=A_yB_z-A_zB_y\\
C_y&=A_zB_x-A_xB_z\\
C_z&=A_xB_y-A_yB_x\\
\end{align*}
\section{Dezplazamiento, tiempo y velocidad media}



\end{document}
