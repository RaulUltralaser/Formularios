\documentclass[twocolumn]{article}
\usepackage[utf8]{inputenc}
\usepackage{amsmath}
\setcounter{secnumdepth}{0}
\setcounter{MaxMatrixCols}{20}
\usepackage{graphicx}
\usepackage{float}
\usepackage{caption}
\usepackage{subfig}
\usepackage{hyperref}
\graphicspath{ {./images/} }  

\title{Formulario, Física}
\author{Raúl Ultralaser}
\date{}

\begin{document}

\maketitle

\section{Vectores}
\begin{align*}
\vec{A}&=A_x\hat{i}+A_y\hat{j}+A_z\hat{k}\\
\vec{B}&=B_x\hat{i}+B_y\hat{j}+B_z\hat{k}\\
\end{align*}
Producto escalar (punto)
\begin{equation*}
    \Vec{A}\cdot \Vec{B}=ABcos\phi=|A||B|cos\phi
\end{equation*}
en términos de sus componentes
\begin{equation*}
    \Vec{A}\cdot \Vec{B}=A_xB_x+A_yB_y+A_zB_z
\end{equation*}
Producto cruz
\begin{equation*}
    \Vec{C}=\Vec{A}\times\Vec{B}=ABsen\phi
\end{equation*}
en términos de sus componentes
\begin{align*}
C_x&=A_yB_z-A_zB_y\\
C_y&=A_zB_x-A_xB_z\\
C_z&=A_xB_y-A_yB_x\\
\end{align*}
\section{Movimiento rectilíneo}

Velocidad meda $x$, movimiento rectilíneo
\begin{equation*}
    v_{med-x}=\frac{x_2-x_1}{t_2-t_1}=\frac{\Delta x}{\Delta t}
\end{equation*}
Componente $x$ de la velocidad instantánea.
\begin{equation*}
    v_x=\lim_{\Delta t\rightarrow 0}\frac{\Delta x}{\Delta t}=\frac{dx}{dt}
\end{equation*}
Aceleración media, movimiento rectilíneo,
\begin{equation*}
    a_{med-x}=\frac{v_{2x}-v_{1x}}{t_2-t_1}=\frac{\Delta v_x}{\Delta t}
\end{equation*}
aceleración instantánea, movimiento rectilíneo.
\begin{equation*}
    a_x=\lim_{\Delta t\rightarrow 0}\frac{\Delta v_x}{\Delta t}=\frac{dv_x}{dt}
\end{equation*}
Movimiento con aceleración constante (solo con aceleración constante)
\begin{align*}
    v_x&=v_{0x}+a_xt\\
    x&=x_0+v_{0x}t+\frac{1}{2}a_xt^2\\
    v_x^2&=v^2_{0x}+2a_x(x-x_0)\\
    x-x_0&=\left( \frac{v_{0x}+v_x}{2} \right)t.
\end{align*}
Velocidad y posición por integración
\begin{align*}
    v_x&=v_{0x}+\int_0^ta_xdt\\
    x&=x_0+\int_0^tv_xdt.
\end{align*}
\section{Movimiento en dos o en  tres  dimensiones}

\end{document}
