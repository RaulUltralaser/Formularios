\begin{teo}[\textbf{First isomorphism theorem}]
    Let $phi:G\rightarrow G'$ be a homomorphism of groups. Then
    \begin{equation*}
        G/Ker\phi \simeq Im \phi
    \end{equation*}
    Hence, in particular, if $\phi$ is surjective, then 
    \begin{equation*}
        G/Ker \phi \simeq G'
    \end{equation*}
\end{teo}
\begin{coro}
Any homomorphism $\phi:G\rightarrow G'$ of groups can be factored as 
\begin{equation*}
    \phi=j\cdot\psi\cdot\eta
\end{equation*}
where $\eta:G\rightarrow G/Ker\phi$ is the natural homomorphism, $\psi:G/Ker\phi\rightarrow Im\phi$ is the isomorphism obtained in the theorem, and $j:Im\phi\rightarrow G'$ is the inclusion map.
\begin{center}
    \xymatrix{
    G \ar[r]^{\phi} \ar[d]^{\eta}& G'  \\
    G/Ker\phi \ar[r]^{\phi} & Im\phi \ar[u]^j}
\end{center}
\end{coro}
\begin{teo}[\textbf{Second isomorphism theorem}]
    Let $H$ and $N$ be subgroups of $G$, and $N\vartriangleleft G$. Then
    \begin{equation*}
        H/H\cap N \simeq HN/N
    \end{equation*}
\end{teo}
The inclusion diagram shown below is helpful in visualizing the theorem. Because of this, the theorem is known as the "diamond isomorphism theorem".
\begin{center}
    \xymatrix{
      & G \ar@{-}[d]&\\
      & HN  \ar@{-}[ld] \ar@{-}[rd] &    \\
       H \ar@{-}[rd]& & N\ar@{-}[ld] \\
       &H\cap N&    }
\end{center}
\begin{teo}[\textbf{Third isomorphism theorem}]
    Let $H$ and $K$ be normal subgroups of $G$ and $K\subset H$. Then
    \begin{equation*}
        (G/K)(H/K)\simeq G/H
    \end{equation*}
\end{teo}
This theorem is also known as the "double quotient isomorphism theorem".
\begin{teo}
    Let $G_1$ and $G_2$ be groups, and $N_1\vartriangleleft G_1, N_2\vartriangleleft G_2$. Then $(G_1\times G_2)/(N_1\times N_2)\simeq (G_1/N_1)\times(G_/N_2)$.
\end{teo}
\begin{teo}[\textbf{correspondence theorem}]
    Let $\phi:G\rightarrow G'$ be a homomorphism of a group $G$ onto a group $G'$. Then the following are true:
    \begin{align*}
    (i)&\hspace{.3cm}H<G\Rightarrow \phi(H)<G'.\\
    (i)'&\hspace{.3cm}H'<G'\Rightarrow \phi^{-1}(H')<G.\\
    (ii)&\hspace{.3cm}H\vartriangleleft G\Rightarrow \phi(H)\vartriangleleft G'\\
    (ii)'&\hspace{.3cm}H'\vartriangleleft G'\Rightarrow \phi^{-1}(H')\vartriangleleft G\\
    (iii)&\hspace{.3cm}H<G\hspace{.1cm}and\hspace{.1cm}H\supset Ker\phi \Rightarrow H=\phi^{-1}(\phi(H))\\
    (iv)&\hspace{.3cm}The\hspace{.1cm}maping\hspace{.1cm}H\mapsto\phi(H)\hspace{.1cm}is\hspace{.1cm}a\hspace{.1cm}1-1\hspace{.1cm}correspondence\\
    &\hspace{.1cm}between\hspace{.1cm}the\hspace{.1cm}family\hspace{.1cm}of\hspace{.1cm}subgroups\hspace{.1cm}of\hspace{.1cm}G';futhermore,\\
    &\hspace{.1cm}normal\hspace{.1cm}subgroups\hspace{.1cm}of\hspace{.1cm}G\hspace{.1cm}correspond\hspace{.1cm}to\hspace{.1cm}normal\\
    &\hspace{.1cm}subgroups\hspace{.1cm}of\hspace{.1cm}G'.
    \end{align*}
\end{teo}
\begin{coro}
    Let $N$ be a normal subgroup of $G$. Given any subgroup $H'$ of $G/N$, there is a unique subgroup $H$ of $G$ such that $H'=H/N$. Further, $H\vartriangleleft G$ if and only if $H/N\vartriangleleft G/N$.
\end{coro}
\begin{defi}
 Let $G$ be a group. A normal subgroup $N$ of $G$ is called a maximal normal subgroup if
 \begin{align*}
     (i)&\hspace{.3cm}N\neq G\\
     (ii)&\hspace{.3cm}H\vartriangleleft G\hspace{.1cm}and\hspace{.1cm}H\supset N\Rightarrow H=N\hspace{.1cm}or\hspace{.1cm}H=G.
 \end{align*}
\end{defi}
\begin{defi}
    A group of $G$ is said  to be simple if $G$ has no proper normal subgroups;that is, $G$ has no normal subgroups except $(e)$ and $G$.
\end{defi}
\begin{coro}
    Let $N$ be a proper normal subgroup of $G$. Then $N$ is a maximal normal subgroup of $G$ if and only if $G/N$ is simple.
\end{coro}
\begin{coro}
    Let $H$ and $K$ be a distinct maximal normal subgroups of $G$. Then $H\cap K$ is a maximal normal subgroup of $H$ and also of $K$.
\end{coro}
