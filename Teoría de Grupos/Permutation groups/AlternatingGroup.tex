\begin{teo}
    If a permutation $\sigma\in S_n$ is a product of $r$ transpositions and also a product of $s$ transpositions, then $r$ and $s$ are either both even or both odd.
\end{teo}

\begin{defi}
    A permutation in $S_n$ is called an even (odd) permutation if it is a product of an even (odd) number of transpositions.
\end{defi}
\begin{defi}
    Let $\phi:n\rightarrow n$. Then
    \begin{equation*}
        f(x)= \left\{ \begin{array}{lcc}
             +1 & if\hspace{.1cm}\phi\hspace{.1cm}is\hspace{.1cm}an\hspace{.1cm}even\hspace{.1cm}permutation,  \\
             \\ -1 &  if\hspace{.1cm}\phi\hspace{.1cm}is\hspace{.1cm}an\hspace{.1cm}odd\hspace{.1cm}permutation, \\
             \\0 & if\hspace{.1cm}\phi\hspace{.1cm}is\hspace{.1cm}not\hspace{.1cm}a\hspace{.1cm}permutation,
             \end{array}
   \right.
    \end{equation*}
\end{defi}
\begin{lema}
    Let $\phi,\psi$ be mappings form $n$ to $n$. Then 
    \begin{equation*}
        \upepsilon(\phi\psi)=\upepsilon(\phi)\upepsilon(\psi).
    \end{equation*}
    Hence for any $\sigma\in S_n, \upepsilon(\sigma^{-1})=\upepsilon(\sigma)$.
\end{lema}
\begin{defi}
    The subgroup $A_n$ of all even permutations in $S_n$ is called the alternating group of degree $n$.
\end{defi}