\begin{defi}
Let $(G,\cdot)$ be a group and let $H$ be a subset of $G$. $H$ is called a subgroup of $G$, written $H<G$, if $H$ is a group relative to the binary operation in $G$.
\end{defi}

\begin{teo}
Let $G$ be a group. A nonempty subset $H$ of $G$ is a subgroup of $G$ if and only if either of the following holds:
\begin{align*}
    (i)&\hspace{.3cm}For\hspace{.1cm}all\hspace{.1cm}a,b\in H, ab\in H,\hspace{.1cm}and\hspace{.1cm}a^{-1}\in H.\\
    (ii)&\hspace{.3cm}For\hspace{.1cm}all\hspace{.1cm}ab\in H, ab^{-1}\in H.
\end{align*}
\end{teo}
\begin{teo}
    Let $(G,\cdot)$ be a group. A nonempty finite subset $H$ of $G$ is a subgroup if and only if $ab\in H$ for all $a.b\in H$
\end{teo}
\begin{teo}
    Let $\phi:G\rightarrow H$ be a homomorphism of groups. Then $Ker\phi$ is a subgroup of $G$ and $Im\phi$ is a subgroup of $H$.
\end{teo}
\begin{defi}
    The center of a group $G$, written $Z(G)$, is the set of those elements in $G$ that commute with every element in $G$; that is,
    \begin{equation*}
        Z(G)=\{a\in G|ax=xa\hspace{.1cm}for\hspace{.1cm}all\hspace{.1cm}x\in G\}
    \end{equation*}
\end{defi}
\begin{teo}
    The center of a group $G$ is a subgroup of $G$
\end{teo}
\begin{teo}
    Let $H$ and $K$ be subgroups of a group $(G,\cdot)$. Then $HK$ is a subgroup of $G$ if and only if $HK=KH$.
\end{teo}
\begin{teo}
    Let $S$ be a nonempty subset of a group $G$. Then the subgroup generated by $S$ is the set $M$ of all finite products $x_1,x_2,...,x_n$ such that, for each $i$, $x_i\in S$ or $x_i^{-1}\in S$
\end{teo}
\begin{teo}
Let $G$ be a group and $a\in G$
\begin{align*}
    (i)&\hspace{.3cm}If\hspace{.1cm}a^n=e\hspace{.1cm}for\hspace{.1cm}some\hspace{.1cm}integer\hspace{.1cm}n\neq 0,\hspace{.1cm}then\hspace{.1cm}o(a)|n\\
    (ii)&\hspace{.3cm}If\hspace{.1cm}o(a)=m\hspace{.1cm}then\hspace{.1cm}for\hspace{.1cm}all\hspace{.1cm}integers\hspace{.1cm}i,a^i=a^{r(i)},\\
    \hspace{.1cm}&where\hspace{.1cm} r(i)\hspace{.1cm}is\hspace{.1cm}the\hspace{.1cm}remainder\hspace{.1cm}of\hspace{.1cm}i\hspace{.1cm}modulo\hspace{.1cm}m.\\
    (iii)&\hspace{.3cm}[a]\hspace{.1cm}is\hspace{.1cm}of\hspace{.1cm}order\hspace{.1cm}m\hspace{.1cm}if\hspace{.1cm}and\hspace{.1cm}only\hspace{.1cm}if\hspace{.1cm}o(a)=m.
\end{align*}
\end{teo}
\begin{coro}
If $G$ is a finite group, then there exist a positive integer $k$ such that $x^k=e$ for all $x\in G$.
\end{coro}
\begin{defi}
    Let $H$ be a subgroup of $G$. Given $a\in G$, the set
    \begin{equation*}
        aH=\{ah|h\in H\}
    \end{equation*}
    is called the left coset of $H$ determined by $a$. A subset $C$ of $G$ is called a left coset of $H$ in $G$ if $C=aH$ for some $a$ in $G$. The set of all left cosets of $H$ in $G$ is written $G/H$
\end{defi}
\begin{defi}
    Let $H$ be a subgroup of $G$. The cardinal number of the set of left (right) cosets of $H$ in $G$ is called the index of $H$ in $G$ and denoted by $[G:H]$.
\end{defi}
\begin{teo}[\textbf{Lagrange}]
Let $G$ be a finite group. Then the order of any subgroup of $G$ divides the order of $G$.
\end{teo}
\begin{coro}
    Let $G$ be a finite group of order $n$. Then for every $a\in G$, $o(a)|n$, and, hence, $a^n=e$.

    Consequently, every finite group of prime order is cyclic and, hence, abelian.
\end{coro}