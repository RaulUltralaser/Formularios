The simplest algebraic structure to recognize is a semigroup, which is defined as a nonempty set $S$ with an associative binary operation.  

\begin{defi}
Let $(S,\cdot)$ be a semigroup. If there is an element $e$, in $S$ such that \begin{equation*}
    ex=x=xe \hspace{1cm} for \hspace{.2cm} all \hspace{.2cm} x\in S,
\end{equation*}
then $e$ is called the identity of the semigroup $(S,\cdot)$.
\end{defi}
\begin{defi}
    Let $(S,\cdot)$ be a semigroup with identity $e$. Let $a\in S$. If there exist an element $b$ in $S$ such that \begin{equation*}
        ab=e=ba
    \end{equation*}then b is called the inverse of $a$, and $a$ is said to be invertible
\end{defi}
\begin{defi}
    A nonempty set $G$ with a binary operation $\cdot$ on $G$ is called a group if the following axioms hold:
    \begin{align*}
        (i)&\hspace{.3cm} a(bc)=(ab)c \hspace{.1cm}for\hspace{.1cm}all\hspace{.1cm}a,b,c\in G.\\
        (ii)&\hspace{.3cm}There\hspace{.1cm}exist\hspace{.1cm}e\in G \hspace{.1cm}such\hspace{.1cm} that\hspace{.1cm} ea=a\hspace{.1cm}\\ &for\hspace{.1cm}all\hspace{.1cm}a\in G.\\
        (iii)&\hspace{.3cm}For\hspace{.1cm}every\hspace{.1cm}a\in G\hspace{.1cm}\\
        &there\hspace{.1cm}exist\hspace{.1cm}a'\in G\hspace{.1cm}such\hspace{.1cm}that\hspace{.1cm}a'a=e
    \end{align*}
\end{defi}

\begin{teo}
A semigroup $G$ is a group if and only if for all $a,b$ in $G$, each of the equations $ax=b$ and $ya=b$ has a solution.
\end{teo}

\begin{teo}
    A finite semigroup $G$ is a group if and only if the cancelation laws hold for all elements in $G$; that is,
    \begin{equation*}
        ab=ac\Rightarrow b=c \hspace{.3cm} and \hspace{.3cm} ba=ca\Rightarrow b=c
    \end{equation*}
    for all $a,b,c\in G$
\end{teo}