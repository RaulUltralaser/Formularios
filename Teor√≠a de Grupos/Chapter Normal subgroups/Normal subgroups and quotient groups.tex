\begin{defi}
    Let $G$ be a group. A subgroup $N$ of $G$ is called a normal subgroup of $G$, written $N\vartriangleleft G$, if $xNx^{-1}\subset N$ for every $x\in G$.
\end{defi}
\begin{teo}
Let $N$ be a subgroup of a group $G$. Then the following are equivalent.
\begin{align*}
    (i)&\hspace{.3cm}N\vartriangleleft G\\
    (ii)&\hspace{.3cm}xNx^{-1}=N\hspace{.1cm}for\hspace{.1cm}every\hspace{.1cm}x\in G\\
    (iii)&\hspace{.3cm}xN=Nx\hspace{.1cm}for\hspace{.1cm}every\hspace{.1cm}x\in G\\
    (iv)&\hspace{.3cm}(xN)(yN)=xyN\hspace{.1cm}for\hspace{.1cm}all\hspace{.1cm}x,y\in G.
\end{align*}
\end{teo}
\begin{teo}
    Let $N$ be a normal subgroup of the group $G$. Then $G/N$ is a group under multiplication. The mapping $\phi:G\rightarrow G/N$, given by $x\mapsto xN$, is a surjective homomorphism, and $Ker\phi=N$
\end{teo}
\begin{defi}
    Let $N$ be a normal subgroup of $G$. The group $G/N$ is called the quotient group of $G$ by $N$. The homomorphism $G\rightarrow G/N$, given by $x\mapsto xN$, is called the natural (or canonical) homomorphism of $G$ onto $G/N$.
\end{defi}
\begin{defi}
    Let $G$ be a group, and let $S$ be a nonempty subset of $G$. The normalizer of $S$ in $G$ is the set 
    \begin{equation*}
        N(S)=\{x\in G|xSx^{-1}=S\}
    \end{equation*}
    The normalizer of a singleton $\{a\}$ is written $N(a)$.
\end{defi}
\begin{teo}
    Let $G$ be a group. For any nonempty subset $S$ of $G$, $N(S)$ is a subgroup of $G$. Further, for any subgroup $H$ of $G$,
    \begin{align*}
    (i)&\hspace{.3cm}N(H)\hspace{.1cm}is\hspace{.1cm}the\hspace{.1cm}largest\hspace{.1cm}subgroup\hspace{.1cm}of\hspace{.1cm}G\hspace{.1cm}in\hspace{.1cm}which\hspace{.1cm}H\\
    &\hspace{.1cm}is\hspace{.1cm}normal;\\
    (ii)&\hspace{.3cm}if\hspace{.1cm}K\hspace{.1cm}is\hspace{.1cm}a\hspace{.1cm}subgroup\hspace{.1cm}of\hspace{.1cm}N(H)\hspace{.1cm},then\hspace{.1cm}H\hspace{.1cm}is\hspace{.1cm}a\\&\hspace{.1cm}normal\hspace{.1cm}subgroup\hspace{.1cm}of\hspace{.1cm}KH.
\end{align*}
\end{teo}
\begin{defi}
    Let $G$ be a group. For any $a,b\in G$, $aba^{-1}b^{-1}$ is called a commutator in $G$. The subgroup of $G$ generated by the set of all commutators in $G$ is called the commutator subgroup of $   G$ (or the derived group of $G$) and denoted by $G'$
\end{defi}
\begin{teo}
    Let $G$ be a group, and let $G'$ be the derived of $G$. Then
    \begin{align*}
    (i)&\hspace{.3cm}G'\vartriangleleft G\\
    (ii)&\hspace{.3cm}G/G' is abelian\\
    (iii)&\hspace{.3cm}if\hspace{.1cm}H\vartriangleleft G,\hspace{.1cm} then\hspace{.1cm} G/H \hspace{.1cm}is \hspace{.1cm}abelian\\
    &\hspace{.1cm} if\hspace{.1cm} and\hspace{.1cm} only \hspace{.1cm}if\hspace{.1cm} G'\subset H.
\end{align*}
\end{teo}
